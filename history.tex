\documentclass{article}
\usepackage{ctex}
\usepackage{geometry}
\geometry{a4paper,left=2cm,right=2cm,top=1.5cm,bottom=1.5cm}
\begin{document}
\title{近代史复习}

\author{南京大学~计算机科学与技术系~殷天润~周涛~何润雨}
\maketitle

\section*{南京条约}
\subsection*{时间:1842年}
\subsection*{背景:1840年英国发动鸦片战争}

\subsection*{内容:}
\begin{itemize}
    \item 割地:把香港岛割让给英国。
    \item 赔款:攫取赔款2100万元(银元)。
    \item 开五口:开放广州、厦门、福州、宁波、上海等5个港口城市为通商口岸。
    \item 协定关税:英国商人进出口货物的税率,清政府均“秉公议定则例”。
\end{itemize}
\subsection*{*意义:}
\begin{itemize}
   \item 是近代西方资本主义国家强加在中国人民身上的第一个不平等条约
   \item 英国以武力侵略的方式迫使中国接受其侵略要求,使中国主权国家的独立地位遭到了破坏
   \item 英国强占香港岛,损害了中国领土的完整
   \item 通商口岸成为西方资本主义对中国进行殖民掠夺和不等价交换的中心
   \item 巨额赔款加重了清政府的财政负担,使劳动人民的生活更加艰苦
   \item 《南京条约》签订后,西方列强趁火打劫,相继强迫清政府签订了一系列不平等条约。
   \item 从此中国逐步沦为半殖民地半封建社会。
\end{itemize}
\ \hrule

\section*{马关条约}
\subsection*{时间:1895年}
\begin{itemize}
    \item 割地:割去中国台湾全岛及所有附属各岛屿和澎湖列岛。
    \item 赔款:强迫中国赔款2亿两白银。
    \item 开四口:开沙市、重庆、苏州、杭州为商埠
    \item 允许外国人在中国办工厂。
\end{itemize}
\subsection*{*意义:}
\begin{itemize}
\item 掀起了帝国主义瓜分中国的狂潮
\item 赔款加重了人民的负担
\item 列强势力进一步深入中国内地
\item 阻碍了中国民间资本主义的发展
\item 中国半殖民地化程度大大加深
\end{itemize}
\ \hrule

\section*{辛丑条约}
\subsection*{时间:1901年}
\subsection*{内容:}
\begin{itemize}
    \item 中国支付赔款额4.5亿两白银,分39年还清,本息合计近10亿两之巨。
    \item 强迫清政府作出永远禁止中国人成立或加入任何反对它们的组织的承诺,并规定清政府各级官员如对人民反抗斗争“弹压惩办” 不力,“即行革职,永不叙用”。
    \item *拆毁北京至大沽的炮台,准许各国派兵驻守北京至山海关沿线的铁路
    \item *划定北京东交民巷为“使馆界”,界内不允许中国人居住,由各国派兵保护
    \item *改总理衙门为外务部,位居六部之首
\end{itemize}
\subsection*{*意义:}
\begin{itemize}
    \item 巨额赔款本息合计达白银9.8亿两,给中国人民增加了新的负担
    \item 关于禁止中国人民反侵略和允许帝国主义在华驻军的规定,严重损害了中国主权
    \item 清政府完全成为帝国主义统治中国的工具
    \item 中国完全沦为半殖民地半封建社会
\end{itemize}


\ \hrule

\section*{领事裁判权}
\subsection*{内容:}
\begin{itemize}
    \item 1843年中英《五口通商章程》规定,
    \begin{itemize}
        \item 在通商口岸,中国如与英侨“遇有交涉诉讼”,英国领事有“查察”、“听诉”之权。
        \item “其英人如何科罪,由英国议定章程、法律,发给管事官照办”。
    \end{itemize}
    \item 1844年中美《望厦条约》
    \begin{itemize}
        \item 扩大领事裁判权的范围,即所有美国人在华之一切民事、刑事诉讼,“均由本国领事等官询明办理”。。
    \end{itemize}
\end{itemize}
\subsection*{影响:}

\begin{itemize}
	\item 从此,外国人可以在中国横行不法,中国政府却无权干预。
\end{itemize}
\ \hrule

\section*{早期维新思想}
\subsection*{背景:}
\begin{itemize}
	\item 受到鸦片战争失败的强烈刺激,中国官吏和知识分子中少数爱国的有识之士,开始注意了解国际形势,研究国外史地,总结失败教训,寻找救国的道路和御敌的方法
\end{itemize}
\subsection*{时间}
\begin{itemize}
	\item 鸦片战争后
\end{itemize}
\subsection*{人物以及内容}
\begin{itemize}
    \item 林则徐,近代中国睁眼看世界的第一人。领导查禁鸦片和进行抗英斗争,组织人翻译西方书刊。
    \item 魏源,在1843年1月编成《海国图志》,提出了“师夷长技以制夷”的思想,主张学习外国先进的军事和科学技术,以期富国强兵,抵御外国侵略。
    \item 王韬、薛福成、马建忠、郑观应,主张学习西方先进的科学技术,同时也要求吸纳西方的政治、经济学说。他们的共同特点,就是具有比较强烈的反对外国侵略、追求中国独立富强的爱国思想,以及具有一定程度反对封建专制的民主思想。如郑观应在所著《盛世危言》中提出大力发展民族工商业,同西方国家进行“商战”,设立议院,实行“君民共主”制度等主张。
\end{itemize}
\subsection*{意义:}
\begin{itemize}
	\item 这些主张具有重要的思想启蒙的意义。
\end{itemize}
\ \hrule

\section*{天朝田亩制度}
\subsection*{时间:1853年}
\subsection*{人物:洪秀全}
\subsection*{背景:}
\begin{itemize}
	\item 太平天国1853年三月占领南京定为首都,改名为天京;定都天京后,进行了一系列制度建设
\end{itemize}
\subsection*{内容:}
\begin{itemize}
    \item 土地分配:根据“凡天下田,天下人同耕”“无处不均匀”的原则,以户为单位,不论男女,按人口和年龄平均分配土地。
    \item  产品分配:根据“人人不受私,物物归上主”原则,规定每户留足口粮,其余归国库。。
\end{itemize}
\subsection*{目的:}
\begin{itemize}
	\item 建立“有田同耕,有饭同时,有衣同穿,有钱同使,无处不均匀,无人不保暖”的理想社会。实际上是一个以解决土地问题为中心的比较完整的社会改革方案。
\end{itemize}
\subsection*{意义:}
\begin{itemize}
	\item 实质:是太平天国的纲领性文件
    \item  优点:从根本上否定了封建社会的基础即封建土地主的土地所有制,体现了广大农民要求平均分配土地的强烈愿望,是对以往农民战争中“均平富”、“等贵贱”和“均平”、“均田”思想的发展和超越,具有进步意义。
    \item 局限性:它的目标仍然是闭塞的自给自足的自然经济,同时又是一个没有商品交换的和绝对平均的社会,具有空想性质
    \item 实际:实际上平分土地的方案在太平军占领地区也并未付诸实行
\end{itemize}
\ \hrule

\section*{洋务运动的历史作用}
\subsection*{三个方面的内容:}
\begin{itemize}
    \item 经济:洋务派提出``自强''、``求富''的主张,发展军事工业、民用企业,在客观上对中国的早期工业和民族资本主义的发展起了某些促进作用。
    \item 文化:开办新式学堂,是中国近代教育的开始。带来了新的知识:翻译近代自然科学书籍,使人们开阔了眼界。
    \item 社会:有利于资本主义经济的发展,有利于社会风气的改变。传统的``重本抑末''等观念受到冲击,社会风气和价值观念开始变化,工商业者的地位上升。。
\end{itemize}
\subsection*{失败的原因}
\begin{itemize}
	\item 具有封建性,指导思想是“中学为体,西学为用”,目的是维护和巩固中国封建统治
	\item 对外国具有依赖性,清政府在洋务运动中与西方国家签订了一批不平等条约,在外国的政治经济侵略控制中无法真正富强起来
	\item 洋务企业管理具有腐朽性。尽管具有一定的资本主义性质,但是管理依然是封建衙门式的,不讲效益质量低下贪腐严重。
\end{itemize}
\ \hrule

\section*{清末``新政''}
\subsection*{时间:1901年4月}
\subsection*{背景:}
\begin{itemize}
	\item 清政府内外交困,想要摆脱困境。
	\item 外部: 1901年《辛丑条约》签订,标志着慈禧太后为首的清政府彻底放弃抵抗外国侵略者的念头
	\item 内部: 国人对清政府更为失望,国内要求变革的呼声日渐高涨。
\end{itemize}
\subsection*{内容:}
\begin{itemize}
    \item 成立督办政务处,宣布实行``新政''。
    \item 设立商部、学部、巡警部等中央行政机构;
    \item 裁撤绿营,建立新军;
    \item 颁布商法商律,奖励工商;
    \item 鼓励留学,颁布新的学制,并下令从1906年起正式废除科举考试。
    \item 迫于内外压力,清政府又于1906年宣布``预备仿行宪政'',并于1908年颁布了《钦定宪法大纲》,制定了一个仿效日本实行君主立宪的方案,但又规定了9年的预备立宪期限。
\end{itemize}
\subsection*{意义:}
\begin{itemize}
    \item 预备立宪并没有能够挽救清王朝,反而激化了社会矛盾,加重了危机。
    \item 主要原因在于,清政府改革的根本目的是为了延续其反动统治。
\end{itemize}
\ \hrule

\section*{同盟会}
\subsection*{时间:1905年8月20日}
\subsection*{背景}
\begin{itemize}
	\item 随着新兴知识分子的产生,宣传革命的书籍报刊涌现,民主革命思想广为传播;革命团体涌现。
\end{itemize}
\subsection*{人物:孙中山、黄兴、宋教仁}
\subsection*{内容:}
\begin{itemize}
    \item 在日本东京成立中国同盟会,孙中山被公举为总理,黄兴被任命为执行部庶务,实际主持会内日常工作。同盟会以《民报》为机关报,并确定了革命纲领。
    \item 政治纲领:``驱除鞑虏,恢复中华,创立民国,平均地权''
    \item 概括为三大主义,即民族主义,民权主义,民生主义,被称为三民主义。
\end{itemize}
\subsection*{意义:}
这是近代中国第一个领导资产阶级革命的全国性政党,它的成立标志着中国资产阶级民主革命进入了一个新的阶段。
\\\hrule

\section*{新文化运动的口号}
\begin{itemize}
    \item 民主和科学,即所谓拥护``德先生''(Democracy)和``赛先生''(Science)。
     \item 民主,既是指资产阶级民主主义的制度,也是指资产阶级民主主义的思想。
     \item 科学,则有广狭二义:
     \begin{itemize}
     	\item 狭义的是指自然科学
     	\item 广义是指社会科学
     	\item 陈独秀强调要用自然科学一样的科学精神和科学方法来研究社会,詹姆士的实用主义、伯格森的创造进化论和罗素的新唯实主义这类用某些自然科学成果装饰起来的资产阶级唯心主义思想体系,当时在他心目中也被认为是科学
     
     \end{itemize}	
 \item 目标:在中国建立西方式的资产阶级国家
\end{itemize}
\ \hrule

\section*{国共合作}
\subsection*{背景:}
\begin{itemize}
	\item 共产党:
	\begin{itemize}
		 \item 二七惨案(1923.2.7京汉铁路罢工被镇压)后,中国共产党决定采取更为积极的步骤去联合孙中山领导的中国国民党。
		\item 1923年中共三大通过与国民党合作的决定。
	\end{itemize}
    \item 国民党
    \begin{itemize}
    	\item 孙中山在晚年实现了伟大的思想转变,实行联俄、联共、扶助农工三大政策。
    \end{itemize}
\end{itemize}
\subsection*{内容:}
\begin{itemize}
 	\subsubsection*{形成标志:}
   \item 1924年1月,国民党一大的成功召开,标志着第一次国共合作的正式形成。
  \subsubsection*{政治基础:新三民主义}
    
        \item 民族主义中突出了反帝的内容,强调对外实行中华民族的独立,同时主张国内各民族一律平等;
        \item 民权主义中强调了民主权利应“为一般平民所共有”,不应为“少数人所得而私”;
        \item 民生主义概括为“平均地权”和“节约资本”两大原则,并提出要改善工农得生活状况。
  

\end{itemize}
\subsubsection*{具体内容:}
\begin{itemize}
    \item 国共合作创办了黄埔陆军军官学校,为未来的革命战争准备了军事力量的骨干。
    \item 以五卅运动为起点,掀起了全国范围的大革命高潮。在此基础上,进行了胜利的广东战争,统一并巩固了广东革命根据地。
    \item 进行以推翻北洋军阀统治为目标的北伐战争。
\end{itemize}
\subsubsection*{破裂:}
\begin{itemize}
    \item 1927年4月12日,蒋介石在上海发动反共政变,以“清党”为名,在东南各省大规模捕杀共产党员和革命群众。
    \item 同年7月15日,当时任武汉国民政府主席的汪精卫在武汉召开“分共”会议,并在其辖区内对共产党员和革命群众实行搜捕和屠杀。国共合作全面破裂,大革命最终失败
    \subsubsection*{失败原因}
     
        \item 反革命力量的强大,资产阶级发生严重的动摇,统一战线出现剧烈的分化
        \item 蒋介石集团、汪精卫集团先后被帝国主义势力和地主阶级、买办资产阶级拉进反革命营垒里去了
        \item 中国共产党的中央领导机关在大革命的后期犯了以陈独秀为代表的右倾机会主义错误,放弃无产阶级对革命的领导权
        \item 主观方面中国共产党还处于年幼时期,没有经验
    
\end{itemize}
\subsection*{大革命的意义:}
\begin{itemize}
	\item 是对未来胜利革命的演习
	\item 中共开始探索马克思主义中国话的途径,初步提出了无产阶级领导的反帝反封建的新民主主义革命的基本思想,懂得土地革命和掌握革命武装的重要性
	\item 提高了中国人民的组织程度,觉悟程度
\end{itemize}
\ \hrule

\section*{北伐}
\subsection*{时间:1926年7月}
\subsection*{背景:}
\noindent 北洋军阀统治者全国大部分地区,与南方的国民政府相对立,彼此不断地明争暗斗,打倒帝国主义
\subsection*{内容:}
\begin{itemize}
    \item 目标:推翻北洋军阀统治
    \item 对象: 北洋军阀吴佩孚、孙传芳、张作霖
    \item 成果: 
    \begin{itemize}
    	\item 基本上推翻了北洋军阀的统治,革命势力发展到长江流域,迁都武汉;
    \item  推动工农运动发展(建立农民政权、组织工人罢工);
    \item 沉重打击了帝国主义在中国的统治(收回汉口、九江英租界)。
    \end{itemize}
\end{itemize}
\subsection*{意义:}
\begin{itemize}
    \item 随着北伐的胜利进军,中国形成了历史上空前广大的人民解放运动。
    \item 帝国主义、封建主义的统治受到严重的打击。
    \item 1925年至1927年的中国反帝反封建的革命,比之以往任何一次革命,
    包括辛亥革命和五四运动,群众的动员程度更为广泛,斗争的规模更为宏伟,
    革命的社会内涵更为深刻,因此被称作大革命。
\end{itemize}
\ \hrule

\section*{南昌起义}
\subsection*{时间:1927年8月1日}
\subsection*{背景:}
\noindent 中共吸取国民革命失败的教训,认识到掌握军队的重要性。
\subsection*{人物:周恩来、贺龙、叶挺、朱德、刘伯承}
\subsection*{内容:}
\noindent 率领共产党掌握或影响下的北伐军2万多人在南昌举行起义。
\subsection*{意义:}
\begin{itemize}
    \item 打响了武装反抗国民党反动统治的第一枪。
    \item 这是中国共产党独立领导革命战争、创建人民军队和武装夺取政权的开端。
\end{itemize}
\ \hrule

\section*{伪满政权}
\subsection*{时间:1932年}
\subsection*{背景:}
\noindent 1931年日军占领中国东北
\subsection*{内容:}
\begin{itemize}
    \item 
    1932年,日军扶持拼凑伪“满洲国”,以清朝末代皇帝溥仪为“执政”(两年后改称“皇帝”)。
    \item 是日本军国主义侵略势力和中国封建复辟残余势力相结合而催生的一个怪胎。
    
\end{itemize}
\subsection*{影响:}
\noindent 伪``满洲国''在``日满共同防卫''的借口下,确认日本在中国东北的一切权益。中国东北三省成了日本的殖民地。


\ \hrule

\section*{八百壮士}
\subsection*{时间:1937年10月26日}
\subsection*{背景}
\noindent 从1937年卢沟桥事变,到1938年10月广州、武汉失守,中国抗战处于战略防御阶段。
\subsection*{内容:}
\noindent 在淞沪会战中,第八十八师第五二四团团附谢晋元率孤军据守四行仓库,被上海市民誉为“八百壮士”。
\subsection*{意义:}
\noindent 八百壮士孤军抗击日寇的英雄壮举,震撼了全国人心,引起了国内国际的关注和尊敬。
\\\hrule
\ \\ \ \\
\centerline{by \LaTeX}
\end{document}
